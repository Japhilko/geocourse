\documentclass[ignorenonframetext,]{beamer}
\setbeamertemplate{caption}[numbered]
\setbeamertemplate{caption label separator}{: }
\setbeamercolor{caption name}{fg=normal text.fg}
\beamertemplatenavigationsymbolsempty
\usepackage{lmodern}
\usepackage{amssymb,amsmath}
\usepackage{ifxetex,ifluatex}
\usepackage{fixltx2e} % provides \textsubscript
\ifnum 0\ifxetex 1\fi\ifluatex 1\fi=0 % if pdftex
  \usepackage[T1]{fontenc}
  \usepackage[utf8]{inputenc}
\else % if luatex or xelatex
  \ifxetex
    \usepackage{mathspec}
  \else
    \usepackage{fontspec}
  \fi
  \defaultfontfeatures{Ligatures=TeX,Scale=MatchLowercase}
\fi
\usetheme[]{CambridgeUS}
\usecolortheme{beaver}
\usefonttheme{structurebold}
% use upquote if available, for straight quotes in verbatim environments
\IfFileExists{upquote.sty}{\usepackage{upquote}}{}
% use microtype if available
\IfFileExists{microtype.sty}{%
\usepackage{microtype}
\UseMicrotypeSet[protrusion]{basicmath} % disable protrusion for tt fonts
}{}
\newif\ifbibliography
\hypersetup{
            pdftitle={B5 Simple Features},
            pdfauthor={Jan-Philipp Kolb},
            pdfborder={0 0 0},
            breaklinks=true}
\urlstyle{same}  % don't use monospace font for urls
\usepackage{color}
\usepackage{fancyvrb}
\newcommand{\VerbBar}{|}
\newcommand{\VERB}{\Verb[commandchars=\\\{\}]}
\DefineVerbatimEnvironment{Highlighting}{Verbatim}{commandchars=\\\{\}}
% Add ',fontsize=\small' for more characters per line
\usepackage{framed}
\definecolor{shadecolor}{RGB}{248,248,248}
\newenvironment{Shaded}{\begin{snugshade}}{\end{snugshade}}
\newcommand{\KeywordTok}[1]{\textcolor[rgb]{0.13,0.29,0.53}{\textbf{#1}}}
\newcommand{\DataTypeTok}[1]{\textcolor[rgb]{0.13,0.29,0.53}{#1}}
\newcommand{\DecValTok}[1]{\textcolor[rgb]{0.00,0.00,0.81}{#1}}
\newcommand{\BaseNTok}[1]{\textcolor[rgb]{0.00,0.00,0.81}{#1}}
\newcommand{\FloatTok}[1]{\textcolor[rgb]{0.00,0.00,0.81}{#1}}
\newcommand{\ConstantTok}[1]{\textcolor[rgb]{0.00,0.00,0.00}{#1}}
\newcommand{\CharTok}[1]{\textcolor[rgb]{0.31,0.60,0.02}{#1}}
\newcommand{\SpecialCharTok}[1]{\textcolor[rgb]{0.00,0.00,0.00}{#1}}
\newcommand{\StringTok}[1]{\textcolor[rgb]{0.31,0.60,0.02}{#1}}
\newcommand{\VerbatimStringTok}[1]{\textcolor[rgb]{0.31,0.60,0.02}{#1}}
\newcommand{\SpecialStringTok}[1]{\textcolor[rgb]{0.31,0.60,0.02}{#1}}
\newcommand{\ImportTok}[1]{#1}
\newcommand{\CommentTok}[1]{\textcolor[rgb]{0.56,0.35,0.01}{\textit{#1}}}
\newcommand{\DocumentationTok}[1]{\textcolor[rgb]{0.56,0.35,0.01}{\textbf{\textit{#1}}}}
\newcommand{\AnnotationTok}[1]{\textcolor[rgb]{0.56,0.35,0.01}{\textbf{\textit{#1}}}}
\newcommand{\CommentVarTok}[1]{\textcolor[rgb]{0.56,0.35,0.01}{\textbf{\textit{#1}}}}
\newcommand{\OtherTok}[1]{\textcolor[rgb]{0.56,0.35,0.01}{#1}}
\newcommand{\FunctionTok}[1]{\textcolor[rgb]{0.00,0.00,0.00}{#1}}
\newcommand{\VariableTok}[1]{\textcolor[rgb]{0.00,0.00,0.00}{#1}}
\newcommand{\ControlFlowTok}[1]{\textcolor[rgb]{0.13,0.29,0.53}{\textbf{#1}}}
\newcommand{\OperatorTok}[1]{\textcolor[rgb]{0.81,0.36,0.00}{\textbf{#1}}}
\newcommand{\BuiltInTok}[1]{#1}
\newcommand{\ExtensionTok}[1]{#1}
\newcommand{\PreprocessorTok}[1]{\textcolor[rgb]{0.56,0.35,0.01}{\textit{#1}}}
\newcommand{\AttributeTok}[1]{\textcolor[rgb]{0.77,0.63,0.00}{#1}}
\newcommand{\RegionMarkerTok}[1]{#1}
\newcommand{\InformationTok}[1]{\textcolor[rgb]{0.56,0.35,0.01}{\textbf{\textit{#1}}}}
\newcommand{\WarningTok}[1]{\textcolor[rgb]{0.56,0.35,0.01}{\textbf{\textit{#1}}}}
\newcommand{\AlertTok}[1]{\textcolor[rgb]{0.94,0.16,0.16}{#1}}
\newcommand{\ErrorTok}[1]{\textcolor[rgb]{0.64,0.00,0.00}{\textbf{#1}}}
\newcommand{\NormalTok}[1]{#1}
\usepackage{graphicx,grffile}
\makeatletter
\def\maxwidth{\ifdim\Gin@nat@width>\linewidth\linewidth\else\Gin@nat@width\fi}
\def\maxheight{\ifdim\Gin@nat@height>\textheight0.8\textheight\else\Gin@nat@height\fi}
\makeatother
% Scale images if necessary, so that they will not overflow the page
% margins by default, and it is still possible to overwrite the defaults
% using explicit options in \includegraphics[width, height, ...]{}
\setkeys{Gin}{width=\maxwidth,height=\maxheight,keepaspectratio}

% Prevent slide breaks in the middle of a paragraph:
\widowpenalties 1 10000
\raggedbottom

\AtBeginPart{
  \let\insertpartnumber\relax
  \let\partname\relax
  \frame{\partpage}
}
\AtBeginSection{
  \ifbibliography
  \else
    \let\insertsectionnumber\relax
    \let\sectionname\relax
    \frame{\sectionpage}
  \fi
}
\AtBeginSubsection{
  \let\insertsubsectionnumber\relax
  \let\subsectionname\relax
  \frame{\subsectionpage}
}

\setlength{\parindent}{0pt}
\setlength{\parskip}{6pt plus 2pt minus 1pt}
\setlength{\emergencystretch}{3em}  % prevent overfull lines
\providecommand{\tightlist}{%
  \setlength{\itemsep}{0pt}\setlength{\parskip}{0pt}}
\setcounter{secnumdepth}{0}

\title{B5 Simple Features}
\author{Jan-Philipp Kolb}
\date{23 Oktober 2018}

\begin{document}
\frame{\titlepage}

\begin{frame}[fragile]{Das Paket \texttt{sf}}

\begin{quote}
Simple Features for R
\end{quote}

\begin{Shaded}
\begin{Highlighting}[]
\KeywordTok{library}\NormalTok{(sf)}
\end{Highlighting}
\end{Shaded}

\begin{verbatim}
## Linking to GEOS 3.6.1, GDAL 2.2.3, proj.4 4.9.3
\end{verbatim}

\begin{Shaded}
\begin{Highlighting}[]
\KeywordTok{demo}\NormalTok{(sf}\OperatorTok{::}\NormalTok{affine)}
\end{Highlighting}
\end{Shaded}

\end{frame}

\begin{frame}[fragile]{Beispieldaten bekommen}

\begin{Shaded}
\begin{Highlighting}[]
\KeywordTok{library}\NormalTok{(osmdata)}
\end{Highlighting}
\end{Shaded}

\begin{verbatim}
## Data (c) OpenStreetMap contributors, ODbL 1.0. http://www.openstreetmap.org/copyright
\end{verbatim}

\begin{Shaded}
\begin{Highlighting}[]
\NormalTok{bb_poly <-}\StringTok{ }\KeywordTok{getbb}\NormalTok{(}\DataTypeTok{place_name =} \StringTok{"Amsterdam"}\NormalTok{, }
                 \DataTypeTok{format_out =} \StringTok{"polygon"}\NormalTok{)}
\end{Highlighting}
\end{Shaded}

\begin{Shaded}
\begin{Highlighting}[]
\NormalTok{ls <-}\StringTok{ }\KeywordTok{st_multilinestring}\NormalTok{(bb_poly)}
\end{Highlighting}
\end{Shaded}

\begin{Shaded}
\begin{Highlighting}[]
\NormalTok{pol <-}\StringTok{ }\NormalTok{sf}\OperatorTok{::}\KeywordTok{st_polygon}\NormalTok{(bb_poly)}
\KeywordTok{class}\NormalTok{(pol)}
\end{Highlighting}
\end{Shaded}

\begin{verbatim}
## [1] "XY"      "POLYGON" "sfg"
\end{verbatim}

\begin{Shaded}
\begin{Highlighting}[]
\NormalTok{bb_poly_ma <-}\StringTok{ }\KeywordTok{getbb}\NormalTok{(}\DataTypeTok{place_name =} \StringTok{"Mannheim"}\NormalTok{, }
                 \DataTypeTok{format_out =} \StringTok{"polygon"}\NormalTok{)}

\KeywordTok{plot}\NormalTok{(bb_poly_ma)}
\end{Highlighting}
\end{Shaded}

\includegraphics{simplefeatures_files/figure-beamer/unnamed-chunk-7-1.pdf}

\begin{Shaded}
\begin{Highlighting}[]
\NormalTok{bb_poly_hd <-}\StringTok{ }\KeywordTok{getbb}\NormalTok{(}\DataTypeTok{place_name =} \StringTok{"Heidelberg"}\NormalTok{, }
                 \DataTypeTok{format_out =} \StringTok{"polygon"}\NormalTok{)}
\end{Highlighting}
\end{Shaded}

\begin{Shaded}
\begin{Highlighting}[]
\CommentTok{# x <- osmdata_sf(pol)}
\end{Highlighting}
\end{Shaded}

\end{frame}

\begin{frame}[fragile]{\href{https://r-spatial.github.io/sf/articles/sf5.html}{Graphiken
mit \texttt{sf}}}

\begin{Shaded}
\begin{Highlighting}[]
\KeywordTok{demo}\NormalTok{(nc, }\DataTypeTok{ask =} \OtherTok{FALSE}\NormalTok{, }\DataTypeTok{echo =} \OtherTok{FALSE}\NormalTok{)}
\end{Highlighting}
\end{Shaded}

\begin{verbatim}
## Reading layer `nc.gpkg' from data source `D:\Programme\R-3.5.0\library\sf\gpkg\nc.gpkg' using driver `GPKG'
## Simple feature collection with 100 features and 14 fields
## Attribute-geometry relationship: 0 constant, 8 aggregate, 6 identity
## geometry type:  MULTIPOLYGON
## dimension:      XY
## bbox:           xmin: -84.32385 ymin: 33.88199 xmax: -75.45698 ymax: 36.58965
## epsg (SRID):    4267
## proj4string:    +proj=longlat +datum=NAD27 +no_defs
\end{verbatim}

\begin{Shaded}
\begin{Highlighting}[]
\KeywordTok{plot}\NormalTok{(nc)}
\end{Highlighting}
\end{Shaded}

\begin{verbatim}
## Warning: plotting the first 9 out of 14 attributes; use max.plot = 14 to
## plot all
\end{verbatim}

\includegraphics{simplefeatures_files/figure-beamer/unnamed-chunk-12-1.pdf}

\end{frame}

\begin{frame}[fragile]{\href{https://cran.r-project.org/web/packages/sf/vignettes/sf2.html}{Shapefiles
mit \texttt{sf} importieren}}

\begin{Shaded}
\begin{Highlighting}[]
\KeywordTok{setwd}\NormalTok{(data_path)}
\NormalTok{plz <-}\StringTok{ }\KeywordTok{st_read}\NormalTok{(}\StringTok{"post_pl.shp"}\NormalTok{)}
\end{Highlighting}
\end{Shaded}

\end{frame}

\end{document}
