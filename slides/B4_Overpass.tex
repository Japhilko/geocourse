\PassOptionsToPackage{unicode=true}{hyperref} % options for packages loaded elsewhere
\PassOptionsToPackage{hyphens}{url}
%
\documentclass[ignorenonframetext,]{beamer}
\usepackage{pgfpages}
\setbeamertemplate{caption}[numbered]
\setbeamertemplate{caption label separator}{: }
\setbeamercolor{caption name}{fg=normal text.fg}
\beamertemplatenavigationsymbolsempty
\usepackage{lmodern}
\usepackage{amssymb,amsmath}
\usepackage{ifxetex,ifluatex}
\usepackage{fixltx2e} % provides \textsubscript
\ifnum 0\ifxetex 1\fi\ifluatex 1\fi=0 % if pdftex
  \usepackage[T1]{fontenc}
  \usepackage[utf8]{inputenc}
  \usepackage{textcomp} % provides euro and other symbols
\else % if luatex or xelatex
  \usepackage{unicode-math}
  \defaultfontfeatures{Ligatures=TeX,Scale=MatchLowercase}
\fi
\usetheme[]{CambridgeUS}
\usecolortheme{beaver}
\usefonttheme{structurebold}
% use upquote if available, for straight quotes in verbatim environments
\IfFileExists{upquote.sty}{\usepackage{upquote}}{}
% use microtype if available
\IfFileExists{microtype.sty}{%
\usepackage[]{microtype}
\UseMicrotypeSet[protrusion]{basicmath} % disable protrusion for tt fonts
}{}
\IfFileExists{parskip.sty}{%
\usepackage{parskip}
}{% else
\setlength{\parindent}{0pt}
\setlength{\parskip}{6pt plus 2pt minus 1pt}
}
\usepackage{hyperref}
\hypersetup{
            pdftitle={B4 Overpass},
            pdfauthor={Jan-Philipp Kolb},
            pdfborder={0 0 0},
            breaklinks=true}
\urlstyle{same}  % don't use monospace font for urls
\newif\ifbibliography
\usepackage{color}
\usepackage{fancyvrb}
\newcommand{\VerbBar}{|}
\newcommand{\VERB}{\Verb[commandchars=\\\{\}]}
\DefineVerbatimEnvironment{Highlighting}{Verbatim}{commandchars=\\\{\}}
% Add ',fontsize=\small' for more characters per line
\usepackage{framed}
\definecolor{shadecolor}{RGB}{42,33,28}
\newenvironment{Shaded}{\begin{snugshade}}{\end{snugshade}}
\newcommand{\AlertTok}[1]{\textcolor[rgb]{1.00,1.00,0.00}{#1}}
\newcommand{\AnnotationTok}[1]{\textcolor[rgb]{0.00,0.40,1.00}{\textbf{\textit{#1}}}}
\newcommand{\AttributeTok}[1]{\textcolor[rgb]{0.74,0.68,0.62}{#1}}
\newcommand{\BaseNTok}[1]{\textcolor[rgb]{0.27,0.67,0.26}{#1}}
\newcommand{\BuiltInTok}[1]{\textcolor[rgb]{0.74,0.68,0.62}{#1}}
\newcommand{\CharTok}[1]{\textcolor[rgb]{0.02,0.61,0.04}{#1}}
\newcommand{\CommentTok}[1]{\textcolor[rgb]{0.00,0.40,1.00}{\textbf{\textit{#1}}}}
\newcommand{\CommentVarTok}[1]{\textcolor[rgb]{0.74,0.68,0.62}{#1}}
\newcommand{\ConstantTok}[1]{\textcolor[rgb]{0.74,0.68,0.62}{#1}}
\newcommand{\ControlFlowTok}[1]{\textcolor[rgb]{0.26,0.66,0.93}{\textbf{#1}}}
\newcommand{\DataTypeTok}[1]{\textcolor[rgb]{0.74,0.68,0.62}{\underline{#1}}}
\newcommand{\DecValTok}[1]{\textcolor[rgb]{0.27,0.67,0.26}{#1}}
\newcommand{\DocumentationTok}[1]{\textcolor[rgb]{0.00,0.40,1.00}{\textit{#1}}}
\newcommand{\ErrorTok}[1]{\textcolor[rgb]{1.00,1.00,0.00}{\textbf{#1}}}
\newcommand{\ExtensionTok}[1]{\textcolor[rgb]{0.74,0.68,0.62}{#1}}
\newcommand{\FloatTok}[1]{\textcolor[rgb]{0.27,0.67,0.26}{#1}}
\newcommand{\FunctionTok}[1]{\textcolor[rgb]{1.00,0.58,0.35}{\textbf{#1}}}
\newcommand{\ImportTok}[1]{\textcolor[rgb]{0.74,0.68,0.62}{#1}}
\newcommand{\InformationTok}[1]{\textcolor[rgb]{0.00,0.40,1.00}{\textbf{\textit{#1}}}}
\newcommand{\KeywordTok}[1]{\textcolor[rgb]{0.26,0.66,0.93}{\textbf{#1}}}
\newcommand{\NormalTok}[1]{\textcolor[rgb]{0.74,0.68,0.62}{#1}}
\newcommand{\OperatorTok}[1]{\textcolor[rgb]{0.74,0.68,0.62}{#1}}
\newcommand{\OtherTok}[1]{\textcolor[rgb]{0.74,0.68,0.62}{#1}}
\newcommand{\PreprocessorTok}[1]{\textcolor[rgb]{0.74,0.68,0.62}{\textbf{#1}}}
\newcommand{\RegionMarkerTok}[1]{\textcolor[rgb]{0.74,0.68,0.62}{#1}}
\newcommand{\SpecialCharTok}[1]{\textcolor[rgb]{0.02,0.61,0.04}{#1}}
\newcommand{\SpecialStringTok}[1]{\textcolor[rgb]{0.02,0.61,0.04}{#1}}
\newcommand{\StringTok}[1]{\textcolor[rgb]{0.02,0.61,0.04}{#1}}
\newcommand{\VariableTok}[1]{\textcolor[rgb]{0.74,0.68,0.62}{#1}}
\newcommand{\VerbatimStringTok}[1]{\textcolor[rgb]{0.02,0.61,0.04}{#1}}
\newcommand{\WarningTok}[1]{\textcolor[rgb]{1.00,1.00,0.00}{\textbf{#1}}}
\usepackage{longtable,booktabs}
\usepackage{caption}
% These lines are needed to make table captions work with longtable:
\makeatletter
\def\fnum@table{\tablename~\thetable}
\makeatother
\usepackage{graphicx,grffile}
\makeatletter
\def\maxwidth{\ifdim\Gin@nat@width>\linewidth\linewidth\else\Gin@nat@width\fi}
\def\maxheight{\ifdim\Gin@nat@height>\textheight\textheight\else\Gin@nat@height\fi}
\makeatother
% Scale images if necessary, so that they will not overflow the page
% margins by default, and it is still possible to overwrite the defaults
% using explicit options in \includegraphics[width, height, ...]{}
\setkeys{Gin}{width=\maxwidth,height=\maxheight,keepaspectratio}
% Prevent slide breaks in the middle of a paragraph:
\widowpenalties 1 10000
\raggedbottom
\setbeamertemplate{part page}{
\centering
\begin{beamercolorbox}[sep=16pt,center]{part title}
  \usebeamerfont{part title}\insertpart\par
\end{beamercolorbox}
}
\setbeamertemplate{section page}{
\centering
\begin{beamercolorbox}[sep=12pt,center]{part title}
  \usebeamerfont{section title}\insertsection\par
\end{beamercolorbox}
}
\setbeamertemplate{subsection page}{
\centering
\begin{beamercolorbox}[sep=8pt,center]{part title}
  \usebeamerfont{subsection title}\insertsubsection\par
\end{beamercolorbox}
}
\AtBeginPart{
  \frame{\partpage}
}
\AtBeginSection{
  \ifbibliography
  \else
    \frame{\sectionpage}
  \fi
}
\AtBeginSubsection{
  \frame{\subsectionpage}
}
\setlength{\emergencystretch}{3em}  % prevent overfull lines
\providecommand{\tightlist}{%
  \setlength{\itemsep}{0pt}\setlength{\parskip}{0pt}}
\setcounter{secnumdepth}{0}

% set default figure placement to htbp
\makeatletter
\def\fps@figure{htbp}
\makeatother


\title{B4 Overpass}
\author{Jan-Philipp Kolb}
\date{23 Oktober 2018}

\begin{document}
\frame{\titlepage}

\begin{frame}{Themen dieses Abschnitts}
\protect\hypertarget{themen-dieses-abschnitts}{}

\begin{itemize}
\tightlist
\item
  Die
  \href{https://wiki.openstreetmap.org/wiki/Overpass_API}{\textbf{Overpass
  API}} von Roland Olbricht wird vorgestellt.
\item
  Die API \href{https://overpass-turbo.eu/}{\textbf{Overpass Turbo}}
\item
  Wie man die OSM Daten graphisch darstellen kann.
\end{itemize}

\end{frame}

\begin{frame}{Die Overpass API}
\protect\hypertarget{die-overpass-api}{}

\begin{itemize}
\tightlist
\item
  Die von Roland Olbricht geschriebene Overpass API ermöglicht es
  Entwicklern, kleine Auszüge von benutzergenerierten Inhalten von
  Openstreetmap nach vorgegebenen Kriterien herunterzuladen.
\item
  Overpass ist eine read-only API, die durch den Benutzer ausgewählte
  Teile der OSM-Daten bereitstellt.
\item
  Overpass kann als eine Datenbank über das Internet verstanden werden.
\item
  Die API eignet sich besonders gut, wenn man nach ganz speziellen Map
  Features sucht.
\end{itemize}

\end{frame}

\begin{frame}{\href{https://overpass-turbo.eu/}{Overpass Turbo}}
\protect\hypertarget{overpass-turbo}{}

\begin{figure}
\centering
\includegraphics{figure/overpassTurbo.PNG}
\caption{\url{https://overpass-turbo.eu/}}
\end{figure}

\end{frame}

\begin{frame}[fragile]{Query Overpass}
\protect\hypertarget{query-overpass}{}

\begin{itemize}
\tightlist
\item
  In der folgenden Abfrage wird bei Overpass Turbo nach Bars im
  ausgewählten Fenster gesucht.
\end{itemize}

\begin{verbatim}
node
  [amenity=bar]
  ({{bbox}});
out;
\end{verbatim}

\end{frame}

\begin{frame}{Export bei Overpass}
\protect\hypertarget{export-bei-overpass}{}

\includegraphics{figure/OverpassExport.PNG}

\end{frame}

\begin{frame}{Speicherformate}
\protect\hypertarget{speicherformate}{}

\begin{block}{Bei Export von Overpass}

\begin{itemize}
\tightlist
\item
  GeoJSON
\item
  GPX
\item
  KML
\item
  OSM Rohdaten
\end{itemize}

\end{block}

\end{frame}

\begin{frame}[fragile]{Import von Daten}
\protect\hypertarget{import-von-daten}{}

\begin{Shaded}
\begin{Highlighting}[]
\KeywordTok{library}\NormalTok{(XML)}
\NormalTok{dat <-}\StringTok{ }\KeywordTok{xmlParse}\NormalTok{(}\StringTok{"../data/bus_stop_amsterdam.kml"}\NormalTok{)}
\end{Highlighting}
\end{Shaded}

\begin{Shaded}
\begin{Highlighting}[]
\NormalTok{xmltop <-}\StringTok{ }\KeywordTok{xmlRoot}\NormalTok{(dat)}
\NormalTok{xmltop[[}\DecValTok{1}\NormalTok{]][[}\DecValTok{1}\NormalTok{]]}
\end{Highlighting}
\end{Shaded}

\begin{verbatim}
## <name>overpass-turbo.eu export</name>
\end{verbatim}

\end{frame}

\begin{frame}[fragile]{Xpath Abfragesprache}
\protect\hypertarget{xpath-abfragesprache}{}

\begin{block}{Beispiel: \href{https://de.wikipedia.org/wiki/XPath}{xpath
wikipedia}}

\begin{Shaded}
\begin{Highlighting}[]
\KeywordTok{xpathApply}\NormalTok{(dat,}\StringTok{"Document"}\NormalTok{)}
\end{Highlighting}
\end{Shaded}

\begin{verbatim}
## list()
## attr(,"class")
## [1] "XMLNodeSet"
\end{verbatim}

\end{block}

\end{frame}

\begin{frame}[fragile]{JSON importieren}
\protect\hypertarget{json-importieren}{}

\begin{Shaded}
\begin{Highlighting}[]
\KeywordTok{install.packages}\NormalTok{(}\StringTok{"rjson"}\NormalTok{)}
\KeywordTok{library}\NormalTok{(rjson)}
\end{Highlighting}
\end{Shaded}

\begin{Shaded}
\begin{Highlighting}[]
\KeywordTok{library}\NormalTok{(jsonlite)}
\NormalTok{dat<-jsonlite}\OperatorTok{::}\KeywordTok{fromJSON}\NormalTok{(}\StringTok{"../data/amsterdam_busstop.geojson"}\NormalTok{)}
\KeywordTok{typeof}\NormalTok{(dat)}
\end{Highlighting}
\end{Shaded}

\begin{verbatim}
## [1] "list"
\end{verbatim}

\begin{Shaded}
\begin{Highlighting}[]
\KeywordTok{names}\NormalTok{(dat)}
\end{Highlighting}
\end{Shaded}

\begin{verbatim}
## [1] "type"      "generator" "copyright" "timestamp" "features"
\end{verbatim}

\end{frame}

\begin{frame}[fragile]{Wie sehen die Daten aus}
\protect\hypertarget{wie-sehen-die-daten-aus}{}

\begin{Shaded}
\begin{Highlighting}[]
\NormalTok{DT}\OperatorTok{::}\KeywordTok{datatable}\NormalTok{(dat}\OperatorTok{$}\NormalTok{features}\OperatorTok{$}\NormalTok{properties)}
\end{Highlighting}
\end{Shaded}

\includegraphics{figure/amsterdam_busstop_features.PNG}

\end{frame}

\begin{frame}[fragile]{GPX file importieren}
\protect\hypertarget{gpx-file-importieren}{}

\begin{Shaded}
\begin{Highlighting}[]
\KeywordTok{library}\NormalTok{(plotKML)}
\end{Highlighting}
\end{Shaded}

\begin{verbatim}
## plotKML version 0.5-8 (2017-05-12)
\end{verbatim}

\begin{verbatim}
## URL: http://plotkml.r-forge.r-project.org/
\end{verbatim}

\begin{Shaded}
\begin{Highlighting}[]
\NormalTok{dat_gpx <-}\StringTok{ }\KeywordTok{readGPX}\NormalTok{(}\StringTok{"../data/Amsterdam_busstop.gpx"}\NormalTok{)}
\KeywordTok{head}\NormalTok{(dat_gpx}\OperatorTok{$}\NormalTok{waypoints)}
\end{Highlighting}
\end{Shaded}

\begin{verbatim}
##        lon      lat                            name
## 1 4.880870 52.36213                     Leidseplein
## 2 4.891237 52.37438                             Dam
## 3 4.877558 52.36953                    Elandsgracht
## 4 4.900331 52.37670 Centraal Station / Nicolaaskerk
## 5 4.905498 52.37395               Prins Hendrikkade
## 6 4.890181 52.37310                             Dam
##                                                                                                                            desc
## 1                                                      highway=bus_stop\nname=Leidseplein\npublic_transport=platform\nzone=5700
## 2                 cxx:code=57002550\ncxx:id=31843\nhighway=bus_stop\nname=Dam\npublic_transport=platform\nsource=CXX\nzone=5700
## 3                     bus=yes\nhighway=bus_stop\nname=Elandsgracht\npublic_transport=stop_position\nrailway=tram_stop\ntram=yes
## 4                            bench=yes\nbin=yes\nhighway=bus_stop\nname=Centraal Station / Nicolaaskerk\nshelter=yes\nzone=5700
## 5 bench=yes\nbin=yes\nhighway=bus_stop\nname=Prins Hendrikkade\noperator=GVB\npublic_transport=platform\nshelter=yes\nzone=5700
## 6        bus=yes\ncxx:code=57002560\ncxx:id=31844\nhighway=bus_stop\nname=Dam\npublic_transport=platform\nsource=CXX\nzone=5700
##   link
## 1     
## 2     
## 3     
## 4     
## 5     
## 6
\end{verbatim}

\end{frame}

\begin{frame}{Daten verbinden - Beispiel Bäckereien in Berlin}
\protect\hypertarget{daten-verbinden---beispiel-backereien-in-berlin}{}

\begin{itemize}
\tightlist
\item
  Quelle für die folgenden Daten ist:
\end{itemize}

\includegraphics{figure/osm_freieWeltkarte.PNG}

\end{frame}

\begin{frame}[fragile]{OSM als Datenquelle}
\protect\hypertarget{osm-als-datenquelle}{}

\begin{itemize}
\tightlist
\item
  Zum Download habe ich die
  \href{http://wiki.openstreetmap.org/wiki/Overpass_API}{\textbf{Overpass
  API}} verwendet
\end{itemize}

\begin{Shaded}
\begin{Highlighting}[]
\KeywordTok{load}\NormalTok{(}\StringTok{"../data/info_bar_Berlin.RData"}\NormalTok{)}
\end{Highlighting}
\end{Shaded}

\begin{longtable}[]{@{}llllrr@{}}
\toprule
& addr.postcode & addr.street & name & lat & lon\tabularnewline
\midrule
\endhead
79675952 & 13405 & Scharnweberstraße & Albert's & 52.56382 &
13.32885\tabularnewline
86005430 & NA & NA & Newton Bar & 52.51293 & 13.39123\tabularnewline
111644760 & NA & NA & No Limit Shishabar & 52.56556 &
13.32093\tabularnewline
149607257 & NA & NA & en passant & 52.54420 & 13.41298\tabularnewline
248651127 & 10115 & Bergstraße & Z-Bar & 52.52953 &
13.39564\tabularnewline
267780050 & 10405 & Christburger Straße & Immertreu & 52.53637 &
13.42509\tabularnewline
\bottomrule
\end{longtable}

\end{frame}

\begin{frame}[fragile]{Verwendung des Pakets \texttt{gosmd}}
\protect\hypertarget{verwendung-des-pakets-gosmd}{}

\begin{Shaded}
\begin{Highlighting}[]
\NormalTok{devtools}\OperatorTok{::}\KeywordTok{install_github}\NormalTok{(}\StringTok{"Japhilko/gosmd"}\NormalTok{)}
\end{Highlighting}
\end{Shaded}

\begin{Shaded}
\begin{Highlighting}[]
\KeywordTok{library}\NormalTok{(}\StringTok{"gosmd"}\NormalTok{)}
\NormalTok{pg_MA <-}\StringTok{ }\KeywordTok{get_osm_nodes}\NormalTok{(}\DataTypeTok{object=}\StringTok{"leisure=playground"}\NormalTok{,}\StringTok{"Mannheim"}\NormalTok{)}
\NormalTok{pg_MA <-}\StringTok{ }\KeywordTok{extract_osm_nodes}\NormalTok{(pg_MA,}\DataTypeTok{value=}\StringTok{'playground'}\NormalTok{)}
\end{Highlighting}
\end{Shaded}

\end{frame}

\begin{frame}[fragile]{Matching}
\protect\hypertarget{matching}{}

\begin{Shaded}
\begin{Highlighting}[]
\NormalTok{tab_plz <-}\StringTok{ }\KeywordTok{table}\NormalTok{(info_be}\OperatorTok{$}\NormalTok{addr.postcode)}
\end{Highlighting}
\end{Shaded}

\begin{Shaded}
\begin{Highlighting}[]
\KeywordTok{load}\NormalTok{(}\StringTok{"../data/BE.RData"}\NormalTok{)}
\end{Highlighting}
\end{Shaded}

\begin{Shaded}
\begin{Highlighting}[]
\NormalTok{ind <-}\StringTok{ }\KeywordTok{match}\NormalTok{(BE}\OperatorTok{@}\NormalTok{data}\OperatorTok{$}\NormalTok{PLZ99_N,}\KeywordTok{names}\NormalTok{(tab_plz))}
\NormalTok{ind}
\end{Highlighting}
\end{Shaded}

\begin{verbatim}
##   [1]  1  2  3  4  5  6  7  8 NA  9 NA NA NA NA NA 10 11 12 NA 13 14 15 16
##  [24] 17 18 19 20 21 22 23 24 25 NA 26 27 28 29 NA NA NA NA 30 NA 31 32 33
##  [47] 34 35 NA NA 36 37 38 39 40 41 42 43 44 45 46 47 48 49 50 51 NA 52 53
##  [70] NA 54 55 NA NA NA 56 57 58 59 60 NA NA NA NA NA 61 NA NA NA 62 NA NA
##  [93] NA NA NA NA NA NA NA 63 NA NA 64 NA 65 NA NA NA 66 NA NA NA NA 67 NA
## [116] NA 68 NA NA NA NA NA NA NA NA NA NA NA NA NA NA NA NA NA NA NA NA NA
## [139] NA 69 70 NA 71 72 73 74 75 NA 76 NA NA NA NA NA NA NA NA NA NA NA NA
## [162] 77 NA 78 79 NA NA NA NA 80 NA NA NA NA 81 NA 82 83 84 NA NA NA NA NA
## [185] NA NA NA 85 NA NA
\end{verbatim}

\end{frame}

\begin{frame}[fragile]{Daten anspielen}
\protect\hypertarget{daten-anspielen}{}

\begin{Shaded}
\begin{Highlighting}[]
\NormalTok{BE}\OperatorTok{@}\NormalTok{data}\OperatorTok{$}\NormalTok{num_plz <-}\StringTok{ }\NormalTok{tab_plz[ind]}
\end{Highlighting}
\end{Shaded}

\end{frame}

\begin{frame}[fragile]{Eine Karte von Berlin mit dem Paket
\texttt{tmap}}
\protect\hypertarget{eine-karte-von-berlin-mit-dem-paket-tmap}{}

\begin{Shaded}
\begin{Highlighting}[]
\NormalTok{BE}\OperatorTok{@}\NormalTok{data}\OperatorTok{$}\NormalTok{num_plz[}\KeywordTok{is.na}\NormalTok{(BE}\OperatorTok{@}\NormalTok{data}\OperatorTok{$}\NormalTok{num_plz)] <-}\StringTok{ }\DecValTok{0}
\NormalTok{tmap}\OperatorTok{::}\KeywordTok{qtm}\NormalTok{(BE,}\DataTypeTok{fill =} \StringTok{"num_plz"}\NormalTok{)}
\end{Highlighting}
\end{Shaded}

\includegraphics{B4_Overpass_files/figure-beamer/unnamed-chunk-19-1.pdf}

\end{frame}

\begin{frame}[fragile]{Mehr Informationen einbinden}
\protect\hypertarget{mehr-informationen-einbinden}{}

\begin{itemize}
\tightlist
\item
  Der folgende Datensatz ist eine Kombination aus den vorgestellten
  PLZ-Shapefiles und OSM-Daten die über Overpass heruntergeladen wurden:
\end{itemize}

\begin{Shaded}
\begin{Highlighting}[]
\KeywordTok{load}\NormalTok{(}\StringTok{"../data/osmsa_PLZ_14.RData"}\NormalTok{)}
\end{Highlighting}
\end{Shaded}

\includegraphics{figure/osmsa_ex.PNG}

\end{frame}

\begin{frame}[fragile]{OSM-Daten - Bäckereien in Stuttgart}
\protect\hypertarget{osm-daten---backereien-in-stuttgart}{}

\begin{Shaded}
\begin{Highlighting}[]
\NormalTok{tmap}\OperatorTok{::}\KeywordTok{qtm}\NormalTok{(PLZ_SG,}\DataTypeTok{fill=}\StringTok{"bakery"}\NormalTok{)}
\end{Highlighting}
\end{Shaded}

\includegraphics{B4_Overpass_files/figure-beamer/unnamed-chunk-23-1.pdf}

\end{frame}

\begin{frame}[fragile]{\href{http://www.datasciencetoolkit.org/}{Das
R-Paket \texttt{RDSTK}}}
\protect\hypertarget{das-r-paket-rdstk}{}

\includegraphics{figure/DataScienceToolkit.PNG}

\begin{itemize}
\tightlist
\item
  Data Science Toolkit API
\end{itemize}

\begin{Shaded}
\begin{Highlighting}[]
\KeywordTok{library}\NormalTok{(}\StringTok{"RDSTK"}\NormalTok{)}
\end{Highlighting}
\end{Shaded}

\end{frame}

\begin{frame}{Die Daten für Stuttgart}
\protect\hypertarget{die-daten-fur-stuttgart}{}

\begin{longtable}[]{@{}lr@{}}
\toprule
Type\_landcover & Freq\tabularnewline
\midrule
\endhead
Artificial surfaces and associated areas & 26\tabularnewline
Cultivated and managed areas & 8\tabularnewline
Tree Cover, needle-leaved, evergreen & 1\tabularnewline
\bottomrule
\end{longtable}

\end{frame}

\begin{frame}[fragile]{Eine Karte der Flächenbedeckung erstellen}
\protect\hypertarget{eine-karte-der-flachenbedeckung-erstellen}{}

\begin{itemize}
\tightlist
\item
  Daten von
  \href{http://bioval.jrc.ec.europa.eu/products/glc2000/products.php}{European
  Commission Land Resource Management Unit Global Land Cover 2000.}
\end{itemize}

\begin{Shaded}
\begin{Highlighting}[]
\NormalTok{tmap}\OperatorTok{::}\KeywordTok{qtm}\NormalTok{(PLZ_SG,}\DataTypeTok{fill=}\StringTok{"land_cover.value"}\NormalTok{)}
\end{Highlighting}
\end{Shaded}

\includegraphics{B4_Overpass_files/figure-beamer/unnamed-chunk-27-1.pdf}

\end{frame}

\begin{frame}[fragile]{Die Höhe in Stuttgart}
\protect\hypertarget{die-hohe-in-stuttgart}{}

\begin{itemize}
\tightlist
\item
  Daten von \href{http://srtm.csi.cgiar.org/}{\textbf{NASA and the CGIAR
  Consortium for Spatial Information }}
\end{itemize}

\begin{Shaded}
\begin{Highlighting}[]
\NormalTok{tmap}\OperatorTok{::}\KeywordTok{qtm}\NormalTok{(PLZ_SG,}\DataTypeTok{fill=}\StringTok{"elevation.value"}\NormalTok{)}
\end{Highlighting}
\end{Shaded}

\includegraphics{B4_Overpass_files/figure-beamer/unnamed-chunk-28-1.pdf}

\end{frame}

\end{document}
