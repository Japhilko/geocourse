\documentclass[ignorenonframetext,]{beamer}
\setbeamertemplate{caption}[numbered]
\setbeamertemplate{caption label separator}{: }
\setbeamercolor{caption name}{fg=normal text.fg}
\beamertemplatenavigationsymbolsempty
\usepackage{lmodern}
\usepackage{amssymb,amsmath}
\usepackage{ifxetex,ifluatex}
\usepackage{fixltx2e} % provides \textsubscript
\ifnum 0\ifxetex 1\fi\ifluatex 1\fi=0 % if pdftex
  \usepackage[T1]{fontenc}
  \usepackage[utf8]{inputenc}
\else % if luatex or xelatex
  \ifxetex
    \usepackage{mathspec}
  \else
    \usepackage{fontspec}
  \fi
  \defaultfontfeatures{Ligatures=TeX,Scale=MatchLowercase}
\fi
\usetheme[]{CambridgeUS}
\usecolortheme{beaver}
\usefonttheme{structurebold}
% use upquote if available, for straight quotes in verbatim environments
\IfFileExists{upquote.sty}{\usepackage{upquote}}{}
% use microtype if available
\IfFileExists{microtype.sty}{%
\usepackage{microtype}
\UseMicrotypeSet[protrusion]{basicmath} % disable protrusion for tt fonts
}{}
\newif\ifbibliography
\hypersetup{
            pdftitle={B1 - Das Arbeiten mit OSM Daten},
            pdfauthor={Jan-Philipp Kolb},
            pdfborder={0 0 0},
            breaklinks=true}
\urlstyle{same}  % don't use monospace font for urls
\usepackage{color}
\usepackage{fancyvrb}
\newcommand{\VerbBar}{|}
\newcommand{\VERB}{\Verb[commandchars=\\\{\}]}
\DefineVerbatimEnvironment{Highlighting}{Verbatim}{commandchars=\\\{\}}
% Add ',fontsize=\small' for more characters per line
\usepackage{framed}
\definecolor{shadecolor}{RGB}{248,248,248}
\newenvironment{Shaded}{\begin{snugshade}}{\end{snugshade}}
\newcommand{\KeywordTok}[1]{\textcolor[rgb]{0.13,0.29,0.53}{\textbf{#1}}}
\newcommand{\DataTypeTok}[1]{\textcolor[rgb]{0.13,0.29,0.53}{#1}}
\newcommand{\DecValTok}[1]{\textcolor[rgb]{0.00,0.00,0.81}{#1}}
\newcommand{\BaseNTok}[1]{\textcolor[rgb]{0.00,0.00,0.81}{#1}}
\newcommand{\FloatTok}[1]{\textcolor[rgb]{0.00,0.00,0.81}{#1}}
\newcommand{\ConstantTok}[1]{\textcolor[rgb]{0.00,0.00,0.00}{#1}}
\newcommand{\CharTok}[1]{\textcolor[rgb]{0.31,0.60,0.02}{#1}}
\newcommand{\SpecialCharTok}[1]{\textcolor[rgb]{0.00,0.00,0.00}{#1}}
\newcommand{\StringTok}[1]{\textcolor[rgb]{0.31,0.60,0.02}{#1}}
\newcommand{\VerbatimStringTok}[1]{\textcolor[rgb]{0.31,0.60,0.02}{#1}}
\newcommand{\SpecialStringTok}[1]{\textcolor[rgb]{0.31,0.60,0.02}{#1}}
\newcommand{\ImportTok}[1]{#1}
\newcommand{\CommentTok}[1]{\textcolor[rgb]{0.56,0.35,0.01}{\textit{#1}}}
\newcommand{\DocumentationTok}[1]{\textcolor[rgb]{0.56,0.35,0.01}{\textbf{\textit{#1}}}}
\newcommand{\AnnotationTok}[1]{\textcolor[rgb]{0.56,0.35,0.01}{\textbf{\textit{#1}}}}
\newcommand{\CommentVarTok}[1]{\textcolor[rgb]{0.56,0.35,0.01}{\textbf{\textit{#1}}}}
\newcommand{\OtherTok}[1]{\textcolor[rgb]{0.56,0.35,0.01}{#1}}
\newcommand{\FunctionTok}[1]{\textcolor[rgb]{0.00,0.00,0.00}{#1}}
\newcommand{\VariableTok}[1]{\textcolor[rgb]{0.00,0.00,0.00}{#1}}
\newcommand{\ControlFlowTok}[1]{\textcolor[rgb]{0.13,0.29,0.53}{\textbf{#1}}}
\newcommand{\OperatorTok}[1]{\textcolor[rgb]{0.81,0.36,0.00}{\textbf{#1}}}
\newcommand{\BuiltInTok}[1]{#1}
\newcommand{\ExtensionTok}[1]{#1}
\newcommand{\PreprocessorTok}[1]{\textcolor[rgb]{0.56,0.35,0.01}{\textit{#1}}}
\newcommand{\AttributeTok}[1]{\textcolor[rgb]{0.77,0.63,0.00}{#1}}
\newcommand{\RegionMarkerTok}[1]{#1}
\newcommand{\InformationTok}[1]{\textcolor[rgb]{0.56,0.35,0.01}{\textbf{\textit{#1}}}}
\newcommand{\WarningTok}[1]{\textcolor[rgb]{0.56,0.35,0.01}{\textbf{\textit{#1}}}}
\newcommand{\AlertTok}[1]{\textcolor[rgb]{0.94,0.16,0.16}{#1}}
\newcommand{\ErrorTok}[1]{\textcolor[rgb]{0.64,0.00,0.00}{\textbf{#1}}}
\newcommand{\NormalTok}[1]{#1}
\usepackage{longtable,booktabs}
\usepackage{caption}
% These lines are needed to make table captions work with longtable:
\makeatletter
\def\fnum@table{\tablename~\thetable}
\makeatother
\usepackage{graphicx,grffile}
\makeatletter
\def\maxwidth{\ifdim\Gin@nat@width>\linewidth\linewidth\else\Gin@nat@width\fi}
\def\maxheight{\ifdim\Gin@nat@height>\textheight0.8\textheight\else\Gin@nat@height\fi}
\makeatother
% Scale images if necessary, so that they will not overflow the page
% margins by default, and it is still possible to overwrite the defaults
% using explicit options in \includegraphics[width, height, ...]{}
\setkeys{Gin}{width=\maxwidth,height=\maxheight,keepaspectratio}

% Prevent slide breaks in the middle of a paragraph:
\widowpenalties 1 10000
\raggedbottom

\AtBeginPart{
  \let\insertpartnumber\relax
  \let\partname\relax
  \frame{\partpage}
}
\AtBeginSection{
  \ifbibliography
  \else
    \let\insertsectionnumber\relax
    \let\sectionname\relax
    \frame{\sectionpage}
  \fi
}
\AtBeginSubsection{
  \let\insertsubsectionnumber\relax
  \let\subsectionname\relax
  \frame{\subsectionpage}
}

\setlength{\parindent}{0pt}
\setlength{\parskip}{6pt plus 2pt minus 1pt}
\setlength{\emergencystretch}{3em}  % prevent overfull lines
\providecommand{\tightlist}{%
  \setlength{\itemsep}{0pt}\setlength{\parskip}{0pt}}
\setcounter{secnumdepth}{0}

\title{B1 - Das Arbeiten mit OSM Daten}
\author{Jan-Philipp Kolb}
\date{22 Oktober 2018}

\begin{document}
\frame{\titlepage}

\begin{frame}{\href{http://www.openstreetmap.de/}{OpenStreetMap}
Projekt}

\begin{quote}
OpenStreetMap.org ist ein im Jahre 2004 gegründetes internationales
Projekt mit dem Ziel, eine freie Weltkarte zu erschaffen. Dafür sammeln
wir weltweit Daten über Straßen, Eisenbahnen, Flüsse, Wälder, Häuser und
vieles mehr.
\end{quote}

\url{http://www.openstreetmap.de/}

\end{frame}

\begin{frame}{OpenStreetMap}

\begin{quote}
OpenStreetMap (OSM) ist ein kollaboratives Projekt um eine editierbare
Weltkarte zu erzeugen.
\end{quote}

\href{https://en.wikipedia.org/wiki/OpenStreetMap}{\textbf{Wikipedia -
OpenStreetMap}}

\end{frame}

\begin{frame}{\href{http://wiki.openstreetmap.org/wiki/DE:Map_Features}{OSM
Map Features}}

\includegraphics{figure/osm_mapfeatures.png}

\end{frame}

\begin{frame}{Download von OpenStreetMap Daten}

\begin{itemize}
\item
  \url{https://mapzen.com/} - Ausschnitte von OpenStreetMap für einzelne
  Städte (\href{https://mapzen.com/data/metro-extracts/}{metro
  extracts})
\item
  Über Geofabrik lassen sich aktuelle Ausschnitte (auch Shapefiles)
  herunterladen (\url{http://download.geofabrik.de/})
\item
  Kartendaten (\href{http://www.openaprs.net/}{\textbf{openaprs}})
\end{itemize}

\end{frame}

\begin{frame}[fragile]{Bei großen Datenmengen}

\begin{itemize}
\item
  Hier geht es nur um das Herunterladen kleiner Ausschnitte.
\item
  Wenn größere Datenmengen benötigt werden, sollte man eine
  Datenbanklösung finden.
\item
  \href{http://www.postgresql.org/}{PostgreSQL} hat den Vorteil, dass es
  Open-Source ist.
\item
  \href{http://www.postgresql.org/download/windows/}{Download PostreSQL}
\item
  \href{https://datashenanigan.wordpress.com/2015/05/18/getting-started-with-postgresql-in-r/}{Hier}
  ist eine Einführung in PostgreSQL zu finden
\item
  Sehr empfehlenswert: Arbeiten mit pgAdmin III
\item
  Beispiel: um Verknüpfung zu einer Datenbank herzustellen - Doppelklick
  auf den Server in pgAdmin III
\end{itemize}

\begin{block}{PostGIS für PostgreSQL}

\begin{itemize}
\tightlist
\item
  \href{http://postgis.net/install/}{\textbf{Installieren}} der PostGIS
  Erweiterung:
\end{itemize}

\begin{verbatim}
CREATE EXTENSION postgis;
\end{verbatim}

\end{block}

\end{frame}

\begin{frame}[fragile]{Programm zum Import der OSM Daten in PostgreSQL-
osm2pgsql}

\begin{itemize}
\tightlist
\item
  Läuft unter Linux deutlich besser
\item
  so könnte bspw. ein Import in PostgreSQL aussehen:
\end{itemize}

\begin{verbatim}
osm2pgsql -c -d osmBerlin --slim -C  -k  berlin-latest.osm.pbf
\end{verbatim}

\end{frame}

\begin{frame}{Nutze bspw. \href{http://www.qgis.org/de/site/}{QGIS} um
Shapefiles zu extrahieren}

\begin{itemize}
\tightlist
\item
  \href{http://www.qgistutorials.com/de/docs/downloading_osm_data.html}{Plugin
  OpenLayers}
\end{itemize}

\includegraphics{figure/stamen_watercolor1.png}

\end{frame}

\begin{frame}{\href{http://www.openstreetmap.org/export}{OSM Ausschnitte
herunterladen}}

\textless{}www.openstreetmap.org/export\textgreater{}

\includegraphics{figure/openstreetmap_export-1024x505.png}

\end{frame}

\begin{frame}[fragile]{Das R-Paket \texttt{XML} - Gaston Sanchez}

\begin{Shaded}
\begin{Highlighting}[]
\KeywordTok{library}\NormalTok{(}\StringTok{"XML"}\NormalTok{)}
\end{Highlighting}
\end{Shaded}

\begin{block}{Gaston Sanchez - Dataflow}

\includegraphics{figure/GastonSanchez2.png}

Seine Arbeit sieht man \href{http://gastonsanchez.com/}{\textbf{hier}}.

\end{block}

\end{frame}

\begin{frame}{\href{https://github.com/gastonstat/tutorial-R-web-data/blob/master/04-parsing-xml/04-parsing-xml.pdf}{Das
Arbeiten mit XML Daten}}

\includegraphics{figure/GastonSanchez3.PNG}

\end{frame}

\begin{frame}{Funktionen im XML Paket}

\begin{longtable}[]{@{}ll@{}}
\toprule
Function & Description\tabularnewline
\midrule
\endhead
xmlName() & name of the node\tabularnewline
xmlSize() & number of subnodes\tabularnewline
xmlAttrs() & named character vector of all attributes\tabularnewline
xmlGetAttr() & value of a single attribute\tabularnewline
xmlValue() & contents of a leaf node\tabularnewline
xmlParent() & name of parent node\tabularnewline
xmlAncestors() & name of ancestor nodes\tabularnewline
getSibling() & siblings to the right or to the left\tabularnewline
xmlNamespace() & the namespace (if there's one)\tabularnewline
\bottomrule
\end{longtable}

\end{frame}

\begin{frame}{\href{http://www.openstreetmap.org/export}{Einzelne
Objekte finden}}

\textless{}www.openstreetmap.org/export\textgreater{}

\includegraphics{figure/admgrBer.PNG}

\end{frame}

\begin{frame}[fragile]{Beispiel: administrative Grenzen Berlin}

\href{http://wiki.openstreetmap.org/wiki/DE:Grenze\#Bundesl.C3.A4ndergrenze_-_admin_level.3D4}{Administrative
Grenzen für Deutschland}

\begin{Shaded}
\begin{Highlighting}[]
\NormalTok{url <-}\StringTok{ "https://api.openstreetmap.org/api/0.6/relation/62422"}
\end{Highlighting}
\end{Shaded}

\begin{Shaded}
\begin{Highlighting}[]
\NormalTok{BE <-}\StringTok{ }\KeywordTok{xmlParse}\NormalTok{(url)}
\end{Highlighting}
\end{Shaded}

\begin{Shaded}
\begin{Highlighting}[]
\NormalTok{BE <-}\StringTok{ }\KeywordTok{xmlParse}\NormalTok{(}\StringTok{"../data/62422.xml"}\NormalTok{)}
\end{Highlighting}
\end{Shaded}

\includegraphics{figure/ExampleAdmBE.PNG}

\end{frame}

\begin{frame}[fragile]{Das XML analysieren}

\begin{itemize}
\tightlist
\item
  \href{http://www.informit.com/articles/article.aspx?p=2215520}{Tobi
  Bosede - Working with XML Data in R}
\end{itemize}

\begin{Shaded}
\begin{Highlighting}[]
\NormalTok{xmltop =}\StringTok{ }\KeywordTok{xmlRoot}\NormalTok{(BE)}
\KeywordTok{class}\NormalTok{(xmltop)}
\end{Highlighting}
\end{Shaded}

\begin{verbatim}
## [1] "XMLInternalElementNode" "XMLInternalNode"       
## [3] "XMLAbstractNode"
\end{verbatim}

\begin{Shaded}
\begin{Highlighting}[]
\KeywordTok{xmlSize}\NormalTok{(xmltop)}
\end{Highlighting}
\end{Shaded}

\begin{verbatim}
## [1] 1
\end{verbatim}

\begin{Shaded}
\begin{Highlighting}[]
\KeywordTok{xmlSize}\NormalTok{(xmltop[[}\DecValTok{1}\NormalTok{]])}
\end{Highlighting}
\end{Shaded}

\begin{verbatim}
## [1] 337
\end{verbatim}

\end{frame}

\begin{frame}[fragile]{Nutzung von Xpath}

\begin{quote}
\href{https://de.wikipedia.org/wiki/XPath}{Xpath}, the XML Path
Language, is a query language for selecting nodes from an XML document.
\end{quote}

\begin{Shaded}
\begin{Highlighting}[]
\KeywordTok{xpathApply}\NormalTok{(BE,}\StringTok{"//tag[@k = 'population']"}\NormalTok{)}
\end{Highlighting}
\end{Shaded}

\begin{verbatim}
## [[1]]
## <tag k="population" v="3440441"/> 
## 
## attr(,"class")
## [1] "XMLNodeSet"
\end{verbatim}

\end{frame}

\begin{frame}[fragile]{Quelle für die Bevölkerungsgröße}

\begin{Shaded}
\begin{Highlighting}[]
\KeywordTok{xpathApply}\NormalTok{(BE,}\StringTok{"//tag[@k = 'source:population']"}\NormalTok{)}
\end{Highlighting}
\end{Shaded}

\begin{verbatim}
## [[1]]
## <tag k="source:population" v="http://www.statistik-berlin-brandenburg.de/Publikationen/Stat_Berichte/2010/SB_A1-1_A2-4_q01-10_BE.pdf 2010-10-01"/> 
## 
## attr(,"class")
## [1] "XMLNodeSet"
\end{verbatim}

-\href{https://www.statistik-berlin-brandenburg.de/datenbank/inhalt-datenbank.asp}{\textbf{Statistik
Berlin Brandenburg}}

\end{frame}

\begin{frame}[fragile]{Etwas überraschend:}

\begin{Shaded}
\begin{Highlighting}[]
\KeywordTok{xpathApply}\NormalTok{(BE,}\StringTok{"//tag[@k = 'name:ta']"}\NormalTok{)}
\end{Highlighting}
\end{Shaded}

\begin{verbatim}
## [[1]]
## <tag k="name:ta" v="<U+0BAA><U+0BC6><U+0BB0><U+0BCD><U+0BB2><U+0BBF><U+0BA9><U+0BCD>"/> 
## 
## attr(,"class")
## [1] "XMLNodeSet"
\end{verbatim}

\includegraphics{figure/OSMBerta.png}

\end{frame}

\begin{frame}[fragile]{Geographische Region}

\begin{Shaded}
\begin{Highlighting}[]
\NormalTok{region <-}\StringTok{ }\KeywordTok{xpathApply}\NormalTok{(BE,}
  \StringTok{"//tag[@k = 'geographical_region']"}\NormalTok{)}
\CommentTok{# regular expressions}
\NormalTok{region[[}\DecValTok{1}\NormalTok{]]}
\end{Highlighting}
\end{Shaded}

\begin{verbatim}
## <tag k="geographical_region" v="Barnim;Berliner Urstromtal;Teltow;Nauener Platte"/>
\end{verbatim}

\begin{verbatim}
<tag k="geographical_region" 
  v="Barnim;Berliner Urstromtal;
  Teltow;Nauener Platte"/>
\end{verbatim}

\end{frame}

\begin{frame}{Landkreis}

\includegraphics{figure/Barnim.png}

\end{frame}

\begin{frame}[fragile]{Weiteres Beispiel}

\begin{Shaded}
\begin{Highlighting}[]
\NormalTok{url2<-}\StringTok{"http://api.openstreetmap.org/api/0.6/node/25113879"}
\NormalTok{obj2<-}\KeywordTok{xmlParse}\NormalTok{(url2)}
\NormalTok{obj_amenity<-}\KeywordTok{xpathApply}\NormalTok{(obj2,}\StringTok{"//tag[@k = 'amenity']"}\NormalTok{)[[}\DecValTok{1}\NormalTok{]]}
\NormalTok{obj_amenity}
\end{Highlighting}
\end{Shaded}

\begin{verbatim}
## <tag k="amenity" v="university"/>
\end{verbatim}

\end{frame}

\begin{frame}[fragile]{Wikipedia Artikel}

\begin{Shaded}
\begin{Highlighting}[]
\KeywordTok{xpathApply}\NormalTok{(obj2,}\StringTok{"//tag[@k = 'wikipedia']"}\NormalTok{)[[}\DecValTok{1}\NormalTok{]]}
\end{Highlighting}
\end{Shaded}

\begin{verbatim}
## <tag k="wikipedia" v="de:Universität Mannheim"/>
\end{verbatim}

\begin{Shaded}
\begin{Highlighting}[]
\KeywordTok{xpathApply}\NormalTok{(obj2,}\StringTok{"//tag[@k = 'wheelchair']"}\NormalTok{)[[}\DecValTok{1}\NormalTok{]]}
\end{Highlighting}
\end{Shaded}

\begin{Shaded}
\begin{Highlighting}[]
\KeywordTok{xpathApply}\NormalTok{(obj2,}\StringTok{"//tag[@k = 'name']"}\NormalTok{)[[}\DecValTok{1}\NormalTok{]]}
\end{Highlighting}
\end{Shaded}

\end{frame}

\begin{frame}[fragile]{Das C und das A}

\begin{Shaded}
\begin{Highlighting}[]
\NormalTok{url3<-}\StringTok{"http://api.openstreetmap.org/api/0.6/node/303550876"}
\NormalTok{obj3 <-}\StringTok{ }\KeywordTok{xmlParse}\NormalTok{(url3)}
\KeywordTok{xpathApply}\NormalTok{(obj3,}\StringTok{"//tag[@k = 'opening_hours']"}\NormalTok{)[[}\DecValTok{1}\NormalTok{]]}
\end{Highlighting}
\end{Shaded}

\begin{verbatim}
## <tag k="opening_hours" v="Mo-Sa 09:00-20:00; Su,PH off"/>
\end{verbatim}

\end{frame}

\begin{frame}[fragile]{Hin und weg}

\begin{Shaded}
\begin{Highlighting}[]
\NormalTok{url4<-}\StringTok{"http://api.openstreetmap.org/api/0.6/node/25439439"}
\NormalTok{obj4 <-}\StringTok{ }\KeywordTok{xmlParse}\NormalTok{(url4)}
\KeywordTok{xpathApply}\NormalTok{(obj4,}\StringTok{"//tag[@k = 'railway:station_category']"}\NormalTok{)[[}\DecValTok{1}\NormalTok{]]}
\end{Highlighting}
\end{Shaded}

\begin{verbatim}
## <tag k="railway:station_category" v="2"/>
\end{verbatim}

\begin{itemize}
\tightlist
\item
  \href{https://de.wikipedia.org/wiki/Bahnhofskategorie}{\textbf{Wikipedia
  Artikel Bahnhofskategorien}}
\end{itemize}

\includegraphics{figure/Bahnhofskategorien.PNG}

\end{frame}

\begin{frame}[fragile]{Exkurs: Bahnhofskategorien}

\begin{itemize}
\tightlist
\item
  \href{https://cran.r-project.org/web/packages/rvest/index.html}{\textbf{rvest:
  Easily Harvest (Scrape) Web Pages}}
\end{itemize}

\begin{Shaded}
\begin{Highlighting}[]
\KeywordTok{library}\NormalTok{(rvest)}
\NormalTok{bhfkat<-}\KeywordTok{read_html}\NormalTok{(}
  \StringTok{"https://de.wikipedia.org/wiki/Bahnhofskategorie"}\NormalTok{)}
\NormalTok{df_html_bhfkat<-}\KeywordTok{html_table}\NormalTok{(}
  \KeywordTok{html_nodes}\NormalTok{(bhfkat, }\StringTok{"table"}\NormalTok{)[[}\DecValTok{2}\NormalTok{]],}\DataTypeTok{fill =} \OtherTok{TRUE}\NormalTok{)}
\end{Highlighting}
\end{Shaded}

\end{frame}

\begin{frame}{Bahnhofskategorien Übersicht}

\begin{longtable}[]{@{}lllllll@{}}
\toprule
Stufe & Bahnsteigkanten & Bahnsteiglänge{[}Anm 1{]} & Reisende/Tag &
Zughalte/Tag & Service{[}Anm 2{]} & Stufenfreiheit{[}Anm
3{]}\tabularnewline
\midrule
\endhead
(0) & --- & --- & --- & --- & Nein & Nein\tabularnewline
1 & 01 & \textgreater{} 000 bis 090 m & 00.000 bis 00.049 & 000 bis 0010
& Ja & Ja\tabularnewline
2 & 02 & \textgreater{} 090 bis 140 m & 00.050 bis 00.299 & 011 bis 0050
& --- & ---\tabularnewline
3 & 03 bis 04 & \textgreater{} 140 bis 170 m & 00.300 bis 0.0999 & 051
bis 0100 & --- & ---\tabularnewline
4 & 05 bis 09 & \textgreater{} 170 bis 210 m & 01.000 bis 09.999 & 101
bis 0500 & --- & ---\tabularnewline
5 & 10 bis 14 & \textgreater{} 210 bis 280 m & 10.000 bis 49.999 & 501
bis 1000 & --- & ---\tabularnewline
6 & 00i ab 15 & \textgreater{} 280 m bis 000 & 000000 ab 50.000 & 000i
ab 1001 & --- & ---\tabularnewline
Gewichtung & 20~\% & 20~\% & 20~\% & 20~\% & 15~\% & 5~\%\tabularnewline
\bottomrule
\end{longtable}

\end{frame}

\begin{frame}[fragile]{Nur fliegen ist schöner}

\begin{Shaded}
\begin{Highlighting}[]
\NormalTok{url5<-}\StringTok{"http://api.openstreetmap.org/api/0.6/way/162149882"}
\NormalTok{obj5<-}\KeywordTok{xmlParse}\NormalTok{(url5)}
\KeywordTok{xpathApply}\NormalTok{(obj5,}\StringTok{"//tag[@k = 'name']"}\NormalTok{)[[}\DecValTok{1}\NormalTok{]]}
\end{Highlighting}
\end{Shaded}

\begin{verbatim}
## <tag k="name" v="City-Airport Mannheim"/>
\end{verbatim}

\begin{Shaded}
\begin{Highlighting}[]
\KeywordTok{xpathApply}\NormalTok{(obj5,}\StringTok{"//tag[@k = 'website']"}\NormalTok{)[[}\DecValTok{1}\NormalTok{]]}
\end{Highlighting}
\end{Shaded}

\begin{verbatim}
## <tag k="website" v="http://www.flugplatz-mannheim.de/"/>
\end{verbatim}

\begin{Shaded}
\begin{Highlighting}[]
\KeywordTok{xpathApply}\NormalTok{(obj5,}\StringTok{"//tag[@k = 'iata']"}\NormalTok{)[[}\DecValTok{1}\NormalTok{]]}
\end{Highlighting}
\end{Shaded}

\begin{verbatim}
## <tag k="iata" v="MHG"/>
\end{verbatim}

\end{frame}

\begin{frame}{Mehr Beispiele, wie man mit XML Daten umgeht:}

\begin{itemize}
\item
  Deborah Nolan -
  \href{http://www.stat.berkeley.edu/~statcur/Workshop2/Presentations/XML.pdf}{\textbf{Extracting
  data from XML}}
\item
  Duncan Temple Lang -
  \href{http://www.omegahat.net/RSXML/shortIntro.pdf}{\textbf{A Short
  Introduction to the XML package for R}}
\end{itemize}

\begin{block}{Noch mehr Informationen}

\begin{itemize}
\item
  \href{http://www.di.fc.ul.pt/~jpn/r/web/index.html\#parsing-xml}{\textbf{Web
  Daten manipulieren}}
\item
  \href{http://www.w3schools.com/xml/xquery_intro.asp}{\textbf{Tutorial
  zu xquery}}
\item
  \href{http://giventhedata.blogspot.de/2012/06/r-and-web-for-beginners-part-ii-xml-in.html}{\textbf{R
  und das Web (für Anfänger), Teil II: XML und R}}
\item
  Gaston Sanchez -
  \href{http://gastonsanchez.com/Handling_and_Processing_Strings_in_R.pdf}{\textbf{String
  Manipulation}}
\item
  \href{https://www.e-education.psu.edu/geog585/node/738}{\textbf{Nutzung,
  Vor- und Nachteile OSM}}
\item
  \href{http://wiki.openstreetmap.org/wiki/Research}{\textbf{Forschungsprojekte
  im Zusammenhang mit OpenStreetMap}}
\end{itemize}

\end{block}

\end{frame}

\begin{frame}[fragile]{Referenzen}

\begin{Shaded}
\begin{Highlighting}[]
\KeywordTok{citation}\NormalTok{(}\StringTok{"XML"}\NormalTok{)}
\end{Highlighting}
\end{Shaded}

\begin{verbatim}
## 
## To cite package 'XML' in publications use:
## 
##   Duncan Temple Lang and the CRAN Team (2018). XML: Tools for
##   Parsing and Generating XML Within R and S-Plus. R package
##   version 3.98-1.12. https://CRAN.R-project.org/package=XML
## 
## A BibTeX entry for LaTeX users is
## 
##   @Manual{,
##     title = {XML: Tools for Parsing and Generating XML Within R and S-Plus},
##     author = {Duncan Temple Lang and the CRAN Team},
##     year = {2018},
##     note = {R package version 3.98-1.12},
##     url = {https://CRAN.R-project.org/package=XML},
##   }
## 
## ATTENTION: This citation information has been auto-generated from
## the package DESCRIPTION file and may need manual editing, see
## 'help("citation")'.
\end{verbatim}

\end{frame}

\begin{frame}[fragile]{Das neuere Paket}

\begin{Shaded}
\begin{Highlighting}[]
\KeywordTok{citation}\NormalTok{(}\StringTok{"xml2"}\NormalTok{)}
\end{Highlighting}
\end{Shaded}

\begin{verbatim}
## 
## To cite package 'xml2' in publications use:
## 
##   Hadley Wickham, James Hester and Jeroen Ooms (2018). xml2: Parse
##   XML. R package version 1.2.0.
##   https://CRAN.R-project.org/package=xml2
## 
## A BibTeX entry for LaTeX users is
## 
##   @Manual{,
##     title = {xml2: Parse XML},
##     author = {Hadley Wickham and James Hester and Jeroen Ooms},
##     year = {2018},
##     note = {R package version 1.2.0},
##     url = {https://CRAN.R-project.org/package=xml2},
##   }
\end{verbatim}

\end{frame}

\begin{frame}{Links}

\begin{itemize}
\item
  \href{http://wiki.openstreetmap.org/wiki/Downloading_data}{\textbf{Wiki
  zum Downlaod}} von Openstreetmap Daten
\item
  \href{http://blog.openstreetmap.de/}{\textbf{Openstreetmap Blog}}
\item
  Liste möglicher Datenquellen für räumliche Analysen
  (\href{http://wiki.openstreetmap.org/wiki/Potential_Datasources}{weltweit}
  und in
  \href{http://wiki.openstreetmap.org/wiki/DE:Potential_Datasources}{\textbf{Deutschland}}
  )
\item
  \href{http://wiki.openstreetmap.org/wiki/SALB}{\textbf{SALB}} -
  Administrative Grenzen
\end{itemize}

\url{http://wiki.openstreetmap.org/wiki/SALB}

\end{frame}

\end{document}
