\PassOptionsToPackage{unicode=true}{hyperref} % options for packages loaded elsewhere
\PassOptionsToPackage{hyphens}{url}
%
\documentclass[ignorenonframetext,]{beamer}
\usepackage{pgfpages}
\setbeamertemplate{caption}[numbered]
\setbeamertemplate{caption label separator}{: }
\setbeamercolor{caption name}{fg=normal text.fg}
\beamertemplatenavigationsymbolsempty
\usepackage{lmodern}
\usepackage{amssymb,amsmath}
\usepackage{ifxetex,ifluatex}
\usepackage{fixltx2e} % provides \textsubscript
\ifnum 0\ifxetex 1\fi\ifluatex 1\fi=0 % if pdftex
  \usepackage[T1]{fontenc}
  \usepackage[utf8]{inputenc}
  \usepackage{textcomp} % provides euro and other symbols
\else % if luatex or xelatex
  \usepackage{unicode-math}
  \defaultfontfeatures{Ligatures=TeX,Scale=MatchLowercase}
\fi
% use upquote if available, for straight quotes in verbatim environments
\IfFileExists{upquote.sty}{\usepackage{upquote}}{}
% use microtype if available
\IfFileExists{microtype.sty}{%
\usepackage[]{microtype}
\UseMicrotypeSet[protrusion]{basicmath} % disable protrusion for tt fonts
}{}
\IfFileExists{parskip.sty}{%
\usepackage{parskip}
}{% else
\setlength{\parindent}{0pt}
\setlength{\parskip}{6pt plus 2pt minus 1pt}
}
\usepackage{hyperref}
\hypersetup{
            pdftitle={Geodaten downloaden und visualisieren},
            pdfauthor={Jan-Philipp Kolb},
            pdfborder={0 0 0},
            breaklinks=true}
\urlstyle{same}  % don't use monospace font for urls
\newif\ifbibliography
% Prevent slide breaks in the middle of a paragraph:
\widowpenalties 1 10000
\raggedbottom
\setbeamertemplate{part page}{
\centering
\begin{beamercolorbox}[sep=16pt,center]{part title}
  \usebeamerfont{part title}\insertpart\par
\end{beamercolorbox}
}
\setbeamertemplate{section page}{
\centering
\begin{beamercolorbox}[sep=12pt,center]{part title}
  \usebeamerfont{section title}\insertsection\par
\end{beamercolorbox}
}
\setbeamertemplate{subsection page}{
\centering
\begin{beamercolorbox}[sep=8pt,center]{part title}
  \usebeamerfont{subsection title}\insertsubsection\par
\end{beamercolorbox}
}
\AtBeginPart{
  \frame{\partpage}
}
\AtBeginSection{
  \ifbibliography
  \else
    \frame{\sectionpage}
  \fi
}
\AtBeginSubsection{
  \frame{\subsectionpage}
}
\setlength{\emergencystretch}{3em}  % prevent overfull lines
\providecommand{\tightlist}{%
  \setlength{\itemsep}{0pt}\setlength{\parskip}{0pt}}
\setcounter{secnumdepth}{0}

% set default figure placement to htbp
\makeatletter
\def\fps@figure{htbp}
\makeatother


\title{Geodaten downloaden und visualisieren}
\author{Jan-Philipp Kolb}
\date{22 Oktober 2018}

\begin{document}
\frame{\titlepage}

\begin{frame}

Dieser Workshop beschäftigt sich mit der Erfassung und Verarbeitung von
räumlichen Informationen (Geodaten) im wissenschaftlichen Kontext.

\end{frame}

\hypertarget{erste-schritte}{%
\subsection{Erste Schritte}\label{erste-schritte}}

\begin{frame}{\textbf{(A1) Einleitung}}
\protect\hypertarget{a1-einleitung}{}

\begin{itemize}
\tightlist
\item
  Was ist das Ziel dieses Kurses und welche Datenquellen werden wir
  verwenden () \href{slides/Intro.md}{Github} \textbar{}
  \href{slides/Intro.pdf}{pdf})
\item
  \href{http://rpubs.com/Japhilko82/mapFirenze}{Regionale Information}
\end{itemize}

\end{frame}

\begin{frame}[fragile]{\textbf{(A2) Das Paket
\href{http://journal.r-project.org/archive/2013-1/kahle-wickham.pdf}{\texttt{ggmap}}}
zur Erzeugung verschiedener Kartentypen.}
\protect\hypertarget{a2-das-paket-ggmap-zur-erzeugung-verschiedener-kartentypen.}{}

\begin{itemize}
\tightlist
\item
  \href{slides/ggmap.md}{Browser} \textbar{}
  \href{slides/ggmap.pdf}{pdf} \textbar{} \href{slides/ggmap.R}{rcode}
\item
  Aufgabe: \href{exercises/Aufgabe_Nutzung_ggmap.Rmd}{Nutzung von
  \texttt{ggmap}}
\end{itemize}

\end{frame}

\begin{frame}{\textbf{(A3) Thematische Karten mit dem R-Paket
\texttt{tmap}}}
\protect\hypertarget{a3-thematische-karten-mit-dem-r-paket-tmap}{}

\begin{itemize}
\tightlist
\item
  \href{slides/tmap.md}{Browser} \textbar{} \href{slides/tmap.pdf}{pdf}
  \textbar{} \href{slides/tmap.pdf}{pdf} \textbar{}
  \href{rcode/tmap.R}{rcode}
\end{itemize}

\end{frame}

\begin{frame}{\textbf{(A4) Choroplethen erzeugen}}
\protect\hypertarget{a4-choroplethen-erzeugen}{}

\begin{itemize}
\tightlist
\item
  \href{slides/Choroplethen.md}{Github}
\end{itemize}

\end{frame}

\begin{frame}[fragile]{\textbf{(A5) Die Nutzung von Shapefiles}}
\protect\hypertarget{a5-die-nutzung-von-shapefiles}{}

\begin{itemize}
\item
  \href{slides/Shapefiles.md}{Github}
\item
  Aufgabe:
  \href{https://github.com/Japhilko/GeoData/blob/master/2017/tutorial/Aufgabe_Verbindung.Rmd}{Zensus
  Ergebnisse und Karte miteinander verbinden und einfärben}
\item
  Aufgabe:
  \href{https://github.com/Japhilko/GeoData/blob/master/2016/tutorial/Aufgabe_Zensus_Ergebnisse.md}{Deutschlands
  Gemeinden}
\item
  Aufgabe:
  \href{https://github.com/Japhilko/GeoData/blob/master/2016/tutorial/Aufgabe_choroplethr.Rmd}{Darstellung
  von Eurostat Daten mit \texttt{choroplethr}}
\item
  Aufgabe:
  \href{https://github.com/Japhilko/GeoData/blob/master/2016/tutorial/Aufgabe_Zensus_Ergebnisse.md}{Deutschlands
  Gemeinden}
\end{itemize}

\end{frame}

\begin{frame}{\textbf{(A6) Das R-Paket \texttt{spdep} - Nachbarschaft
und Distanz}}
\protect\hypertarget{a6-das-r-paket-spdep---nachbarschaft-und-distanz}{}

\begin{itemize}
\tightlist
\item
  \href{slides/spdep.md}{Browser} \textbar{}
  \href{slides/spdep.pdf}{pdf} \textbar{}
  \href{https://raw.githubusercontent.com/Japhilko/GeoData/master/2016/rcode/slidesH1_spdep.R}{rcode}
\item
  Aufgabe:
  \href{https://github.com/Japhilko/GeoData/blob/master/2016/tutorial/Aufgabe_Distanzberechnung.Rmd}{Distanzberechnung}
\end{itemize}

\end{frame}

\begin{frame}{\textbf{(A7) Rasterdaten importieren und verarbeiten}}
\protect\hypertarget{a7-rasterdaten-importieren-und-verarbeiten}{}

\begin{itemize}
\tightlist
\item
  \href{slides/Rasterdaten.Rmd}{}
\end{itemize}

\end{frame}

\hypertarget{b-das-openstreetmap-projekt-und-komplexere-schritte}{%
\subsection{\texorpdfstring{\textbf{(B) Das OpenStreetMap Projekt und
komplexere
Schritte}}{(B) Das OpenStreetMap Projekt und komplexere Schritte}}\label{b-das-openstreetmap-projekt-und-komplexere-schritte}}

\begin{frame}{\textbf{(B1) Openstreetmap}
\href{slides/OpenStreetMap.md}{Github}}
\protect\hypertarget{b1-openstreetmap-github}{}

\begin{itemize}
\item
  Was ist das Openstreetmap Projekt
\item
  \href{https://github.com/Japhilko/GeoData/blob/master/2017/slides/OpenStreetMap.md}{Browser}\textbar{}
  \href{slides/OpenStreetMap.pdf}{pdf}
\end{itemize}

\end{frame}

\begin{frame}{\textbf{(B2) Geokodierung}}
\protect\hypertarget{b2-geokodierung}{}

\begin{itemize}
\tightlist
\item
  \href{slides/Geokodierung.md}{Github}
\end{itemize}

\end{frame}

\begin{frame}{\textbf{(B3) Das Arbeiten mit OSM API`s}}
\protect\hypertarget{b3-das-arbeiten-mit-osm-apis}{}

\begin{itemize}
\tightlist
\item
  Beispiel \href{slides/osm_mainapi.Rmd}{\emph{main OSM API}}
\item
  Die Nutzung der Overpass API
\item
  \textbf{Beispiel Hostels in Madrid}
  (\href{https://github.com/Japhilko/GeoData/blob/master/2016/slides/Madrid_hostels.Rmd}{Browser}),
\item
  Beispiel: \emph{Points of interest}
  (\href{https://rpossib.wordpress.com/2015/09/15/points-of-interest-for-backpackers/}{poi})
  für Backpacker in Amsterdam
\item
  Beispiel:
  \href{https://rpossib.wordpress.com/2015/11/20/use-openstreetmap-date/}{Energieerzeugung}
\item
  \textbf{Aufgabe:}
  \href{https://github.com/Japhilko/GeoData/blob/master/2016/tutorial/Aufgabe_osmar.Rmd}{Darstellung
  von OSM Daten mit tmap}
\end{itemize}

\end{frame}

\begin{frame}{\textbf{(B4) Das \texttt{osmdata} Paket}}
\protect\hypertarget{b4-das-osmdata-paket}{}

\begin{itemize}
\tightlist
\item
  (\href{slides/osmdata.md}{Github})
\end{itemize}

\end{frame}

\begin{frame}{\textbf{(B4) Interaktive Karten mit Javascript
Bibliotheken}}
\protect\hypertarget{b4-interaktive-karten-mit-javascript-bibliotheken}{}

\begin{itemize}
\tightlist
\item
  \href{slides/using_javascript.md}{Github} \textbar{}
  \href{slides/using_javascript.pdf}{pdf} \textbar{}
  \href{rcode/using_javascript.R}{rcode}
\item
  Beispiel - \href{http://rpubs.com/Japhilko82/Campsites}{Campingplätze}
\end{itemize}

\end{frame}

\begin{frame}{\textbf{(B4) Simple Features}}
\protect\hypertarget{b4-simple-features}{}

\begin{itemize}
\tightlist
\item
  \href{slides/simplefeatures.md}{Github} \textbar{}
  \href{slides/simplefeatures.pdf}{pdf} \textbar{}
  \href{rcode/simplefeatures.R}{rcode}
\end{itemize}

\end{frame}

\end{document}
